Generic operations, guided by meta-data, include equality, difference,
merge, parse, and rendering. Operations use cyclic maps to transform 
cyclic data.

Uniform handling of textual and visual grammars. All grammars are bi-directional, with parsing/rendering of text grammars. Editors are specified by visual grammars. Most object construction, create/delete actions, data binding, querying, etc are generated systematically from the grammar.

Modularity based on mixins/merging of schemas, grammars, and other artifacts.

Management of change



\item \Enso\ represents data as collections of entities with attributes
and connected into a graph via relationships. 
Data is not broken into atomic pieces (objects, records, or tuples), but is
instead managed holistically as cyclic data structures.

\item Data is described by schemas, which have 
metadata about structure, behavior, and intent.

\item Generalized grammars describe how
data is presented in a variety of formats, including
text, diagrams, GUIs, and web pages.
Most presentations are bi-directional.

\item Interactive presentation structures define legal editing operations
to insert, delete, and modify the original structure.
Data binding is implicit and automatic.

\item Generic operations, guided by metadata, perform
complex transformations on arbitrary structures.

\item Transformation on cyclic structures are natural.
The system generalizes attribute grammars.

\item Schemas and presentations are also data, and 
are frequently generated, transformed, and merged.

\item Everything, including the fundamental schemas
in the system, is extensible.

\item Change is encouraged and managed. Changes to
schemas are represented as instances of delta schemas.
Schema changes guide instance upgrades.

\item Everything, including schemas, grammars and
transformations, can be composed and mixed together
to support extreme reuse and feature modularity

\item The entire system is dynamic, but tightly 
constrained by the dynamic metadata. We have 
demonstrated that partial evaluation and supercompilation 
can specialize generic operations to create efficient 
code.

\item Serialization is avoided. Instead every kind of 
data is presented in a language, so that it can be 
versioned and managed.

\item There is a natural mapping to relational databases,
especially since bulk operations on information structures
is fundamental to \Enso. We view query results as 
mini-databases, not as single tables. 

\end{itemize}

